% Options for packages loaded elsewhere
\PassOptionsToPackage{unicode}{hyperref}
\PassOptionsToPackage{hyphens}{url}
%
\documentclass[
  10pt,
]{article}
\usepackage{amsmath,amssymb}
\usepackage{lmodern}
\usepackage{ifxetex,ifluatex}
\ifnum 0\ifxetex 1\fi\ifluatex 1\fi=0 % if pdftex
  \usepackage[T1]{fontenc}
  \usepackage[utf8]{inputenc}
  \usepackage{textcomp} % provide euro and other symbols
\else % if luatex or xetex
  \usepackage{unicode-math}
  \defaultfontfeatures{Scale=MatchLowercase}
  \defaultfontfeatures[\rmfamily]{Ligatures=TeX,Scale=1}
\fi
% Use upquote if available, for straight quotes in verbatim environments
\IfFileExists{upquote.sty}{\usepackage{upquote}}{}
\IfFileExists{microtype.sty}{% use microtype if available
  \usepackage[]{microtype}
  \UseMicrotypeSet[protrusion]{basicmath} % disable protrusion for tt fonts
}{}
\makeatletter
\@ifundefined{KOMAClassName}{% if non-KOMA class
  \IfFileExists{parskip.sty}{%
    \usepackage{parskip}
  }{% else
    \setlength{\parindent}{0pt}
    \setlength{\parskip}{6pt plus 2pt minus 1pt}}
}{% if KOMA class
  \KOMAoptions{parskip=half}}
\makeatother
\usepackage{xcolor}
\IfFileExists{xurl.sty}{\usepackage{xurl}}{} % add URL line breaks if available
\IfFileExists{bookmark.sty}{\usepackage{bookmark}}{\usepackage{hyperref}}
\hypersetup{
  pdftitle={Les verbes irréguliers anglais},
  pdfauthor={Adrien Méli},
  hidelinks,
  pdfcreator={LaTeX via pandoc}}
\urlstyle{same} % disable monospaced font for URLs
\usepackage[margin=1in]{geometry}
\usepackage{longtable,booktabs,array}
\usepackage{calc} % for calculating minipage widths
% Correct order of tables after \paragraph or \subparagraph
\usepackage{etoolbox}
\makeatletter
\patchcmd\longtable{\par}{\if@noskipsec\mbox{}\fi\par}{}{}
\makeatother
% Allow footnotes in longtable head/foot
\IfFileExists{footnotehyper.sty}{\usepackage{footnotehyper}}{\usepackage{footnote}}
\makesavenoteenv{longtable}
\usepackage{graphicx}
\makeatletter
\def\maxwidth{\ifdim\Gin@nat@width>\linewidth\linewidth\else\Gin@nat@width\fi}
\def\maxheight{\ifdim\Gin@nat@height>\textheight\textheight\else\Gin@nat@height\fi}
\makeatother
% Scale images if necessary, so that they will not overflow the page
% margins by default, and it is still possible to overwrite the defaults
% using explicit options in \includegraphics[width, height, ...]{}
\setkeys{Gin}{width=\maxwidth,height=\maxheight,keepaspectratio}
% Set default figure placement to htbp
\makeatletter
\def\fps@figure{htbp}
\makeatother
\usepackage[normalem]{ulem}
% Avoid problems with \sout in headers with hyperref
\pdfstringdefDisableCommands{\renewcommand{\sout}{}}
\setlength{\emergencystretch}{3em} % prevent overfull lines
\providecommand{\tightlist}{%
  \setlength{\itemsep}{0pt}\setlength{\parskip}{0pt}}
\setcounter{secnumdepth}{5}
\usepackage[T1]{fontenc}
\usepackage[utf8]{inputenc}
%\usepackage{fontspec}
\usepackage{fontawesome}
\usepackage{fancyhdr}
\usepackage{tipa}
\usepackage{hyperref}
%\usepackage{booktabs}
\usepackage{titlesec}
\usepackage{multicol}
\usepackage{float}
\usepackage{colortbl}
%\usepackage{tufte-book}
\usepackage{xcolor}
%\usepackage{fourier}
%\usepackage{montserrat}
%\usepackage[french]{babel}
\usepackage[modulo]{lineno}
\usepackage{tikz}
\usepackage{graphicx}
\usepackage[light, sfdefault]{roboto}
%\usepackage[left=1.5cm,right=1.5cm,top=1.5cm,bottom=1.5cm]{geometry}

%\renewcommand*\familydefault{\sfdefault} 
\renewcommand{\rmdefault}{ptm}
%----------------------------------------------------------------------------------------
%	DEFINE COLOURS
%----------------------------------------------------------------------------------------
%\definecolor{darkPrimaryColor}{HTML}{303F9F}
\definecolor{darkPrimaryColor}{HTML}{2c5272}
\definecolor{primaryColor}{HTML}{3F51B5}
\definecolor{lightPrimaryColor}{HTML}{C5CAE9}
\definecolor{textPrimaryColor}{HTML}{FFFFFF}
\definecolor{accentColor}{HTML}{FF5252}
\definecolor{primaryTextColor}{HTML}{212121}
\definecolor{secondaryTextColor}{HTML}{757575}
\definecolor{dividerColor}{HTML}{BDBDBD}

% -----------------------------------------------------------------
% Hyper Setup
% -----------------------------------------------------------------
%\hypersetup{
%    %bookmarks=true,         % show bookmarks bar?
%    unicode=false,          % non-Latin characters in Acrobat�s bookmarks
%    pdftoolbar=true,        % show Acrobat�s toolbar?
%    pdfmenubar=true,        % show Acrobat�s menu?
%    pdffitwindow=false,     % window fit to page when opened
%    pdfstartview={FitH},    % fits the width of the page to the window
%    %pdftitle={},    % title
%    pdfauthor={Adrien Meli},     % author 
%    pdfsubject={Phonological rules},   % subject of the document
%    pdfcreator={Creator},   % creator of the document
%    pdfproducer={Producer}, % producer of the document
%    pdfkeywords={Second Language Acquisition, French, English, phonology}, % list of keywords
%    pdfnewwindow=true,     % links in new window
%    colorlinks=true,       % false: boxed links; true: colored links
%    linkcolor=#145680,          % color of internal links
%    citecolor=black,        % color of links to bibliography
%    filecolor=magenta,      % color of file links
%    urlcolor=blue,           % color of external links
%    bookmarksopen=false,
%    anchorcolor=black,
%    bookmarksnumbered=true,
%    pdfpagemode=UseOutlines,    %None/UseOutlines/UseThumbs/FullScreen
%    linktocpage=true
%}
%\newcommand\myshade{85}
%\colorlet{mylinkcolor}{violet}
%\colorlet{mycitecolor}{YellowOrange}
%\colorlet{myurlcolor}{Aquamarine}

\hypersetup{
  linkcolor  = darkPrimaryColor,
  citecolor  = accentColor,
  urlcolor   = darkPrimaryColor,
  colorlinks = true,
}


% ********************Captions and Hyperreferencing / URL **********************

% Captions: This makes captions of figures use a boldfaced small font.

\usepackage[margin=10pt,font=small,labelfont=bf,labelsep=endash]{caption}

% -----------------------------------------------------------------
% TABLE OF CONTENTS
% -----------------------------------------------------------------
\usepackage[dotinlabels]{titletoc}
\titlecontents{section}[0em] % entries are pushed to the rtight
  {} % code to change the appearance
  {} % section number: increase distance to push to the left
  {\hspace*{3.3em}}
  {\titlerule*[1.9mm]{.}\contentspage}
% remove subsections from TOC
\setcounter{tocdepth}{1}

% multi-line curly brackets


%----------------------------------------------------------------------------------------
%	MODIFY SECTION STYLES
%----------------------------------------------------------------------------------------

\usepackage{titlesec} % Required for modifying sections

%------------------------------------------------
% Section

\titleformat
   {\section} % Section type being modified
   [block] % Shape type, can be: hang, block, display, runin, leftmargin, rightmargin, drop, wrap, frame
   {\bfseries\large} % Format of the whole section
   {\thesection.} % Format of the section label
   {6pt} % Space between the title and label
   %{\titlerule\newline{\thetitle. }} % Code before the label
   {\begin{nolinenumbers}} % Code before the label
   [\end{nolinenumbers}]% after code

\titlespacing{\section}
   {0pt} % left
   {\baselineskip} % before title
   {\baselineskip} % Spacing around section titles, the order is: left, before and after

%------------------------------------------------
% Subsection

\titleformat
   {\subsection}
   [block] % Shape type, can be: hang, block, display, runin, leftmargin, rightmargin, drop, wrap, frame
   {\bfseries }% format
   %{\thesubsection.}% label
   {\thesubsection.}% label
   {6pt}% sep btw label and title
   {\begin{nolinenumbers}} % Code before the label
   [\end{nolinenumbers}]% after code
 
\titlespacing{\subsection}
   {6pt}% left
   {\baselineskip} % before title
   {\baselineskip} % Spacing around section titles, the order is: left, before and after

%------------------------------------------------
% Subsubsection

\titleformat
   {\subsubsection}
   [block] % Shape type, can be: hang, block, display, runin, leftmargin, rightmargin, drop, wrap, frame
   {\itshape}% format
   {}% label
   {6pt}% sep btw label and title
   {\begin{nolinenumbers}} % Code before the label
   [\end{nolinenumbers}]% after code
 
\titlespacing{\subsubsection}
   {12pt}% left
   {\baselineskip} % before title
   {\baselineskip}% after sep

%\renewcommand\thesubsection{(\alph{subsection})}

% from bookdown to create column environments
\newenvironment{columns}[1][]{}{}

\newenvironment{column}[1]{\begin{minipage}{#1}\ignorespaces}{%
\end{minipage}
\ifhmode\unskip\fi
\aftergroup\useignorespacesandallpars}

\def\useignorespacesandallpars#1\ignorespaces\fi{%
#1\fi\ignorespacesandallpars}

\makeatletter
\def\ignorespacesandallpars{%
  \@ifnextchar\par
    {\expandafter\ignorespacesandallpars\@gobble}%
    {}%
}
\makeatother


% -----------------------------------------------------------------
% TITLE
% -----------------------------------------------------------------

\title{Mon super template}
\author{Adrien M\'{e}li\\{\tt adrienmeli@gmail.com}}
\date{\today}


% -----------------------------------------------------------------
% DEFINING HEADERS
% -----------------------------------------------------------------

%\lhead{}
\chead{}
\rhead{\tiny{\today}}
\lfoot{}
\cfoot{\thepage}
%\rfoot{\tiny Adrien M\'{e}li \textcopyright \the\year}
\renewcommand{\headrulewidth}{0.1mm}
\renewcommand{\footrulewidth}{0.1mm}

% --------------------------------------
% TO INTEGRATE THE TITLE.TEX
% -------------------------------------
\setlength{\headheight}{14pt}
\newcommand{\HRule}{\rule{\linewidth}{0.5mm}}
\pagestyle{fancy}
\ifluatex
  \usepackage{selnolig}  % disable illegal ligatures
\fi

\title{Les verbes irréguliers anglais}
\author{Adrien Méli}
\date{February 11, 2021}

\begin{document}
\maketitle

%\pagestyle{fancy}
%\begin{linenumbers}[1]
%  \modulolinenumbers[5]

\hypertarget{principes}{%
\section{Principes}\label{principes}}

Les verbes irréguliers s'apprennent en trois colonnes :

\begin{itemize}
\item
  La \textbf{Base Verbale} (BV)
\item
  Le \textbf{Prétérit}
\item
  Le \textbf{Participe passé}
\end{itemize}

\hypertarget{la-base-verbale-bv}{%
\subsection{La Base Verbale (BV)}\label{la-base-verbale-bv}}

Précédée de \emph{to}, la Base Verbale constitue ce que l'on appelle l'infinitif.

À noter que le \textbf{présent simple}, utilisé pour parler des habitudes et vérités générales, ressemble à s'y méprendre à la BV.
La seule différence visible est à la troisième personne du singulier, où le verbe prend un \emph{\textless-(e)s\textgreater{}}\footnote{Ceci est valable pour tous les verbes sauf \emph{BE} (qui se conjugue entièrement), \emph{HAVE} (qui devient ``\emph{has}''), et les 5 auxiliaires modaux, qui eux sont invariables.}.

\hypertarget{le-pruxe9tuxe9rit}{%
\subsection{Le prétérit}\label{le-pruxe9tuxe9rit}}

Le prétérit est utilisé pour parler du passé.

Des trois formes, le prétérit est la seule qui soit véritablement \textbf{conjuguée}.

Pour rappel, pour qu'une phrase soit correcte, il faut qu'elle ait un groupe nominal sujet qui gouverne un verbe conjugué.

Prenons par exemple le verbe \emph{forget} (\emph{forgot, forgotten}), et ajoutons ``John'' devant :

\begin{enumerate}
\def\labelenumi{\arabic{enumi}.}
\tightlist
\item
  \sout{\color[HTML]{f44336}\emph{John forget} \color{black}}
\item
  \textbf{\color[HTML]{4caf50}\emph{John forgot} \color{black}}
\item
  \sout{\color[HTML]{f44336}\emph{John forgotten} \color{black}}
\end{enumerate}

Des trois suggestions ci-dessus, seule la deuxième, \textbf{\color[HTML]{4caf50}\emph{John forgot} \color{black}} (``John a oublié''), compose une phrase correcte.

\hypertarget{le-participe-passuxe9}{%
\subsection{Le participe passé}\label{le-participe-passuxe9}}

Le participe passé s'emploie avec \emph{HAVE} ou \emph{BE}.

\begin{itemize}
\tightlist
\item
  \textbf{Avec \emph{HAVE} :} Le participe passé permet de renvoyer à une action qui est envisagée comme passée.
  Le fait qu'il ne soit pas conjugué permet de combiner ce renvoi au passé avec l'expression d'hypothèses ou de situations irréelles.

  \begin{itemize}
  \tightlist
  \item
    Exemple : \emph{John must have forgotten} \(\rightarrow\) ``John a dû oublier.''
  \end{itemize}
\item
  \textbf{Avec \emph{BE} :} le participe passé revêt alors un \textbf{\color[HTML]{f44336}sens passif \color{black}}. Le sujet du verbe subit l'action.

  \begin{itemize}
  \tightlist
  \item
    Exemple : \emph{The book was written in 1932} \(\rightarrow\) ``Le livre a été écrit en 1932.''
  \end{itemize}
\end{itemize}

\begin{center}\rule{0.5\linewidth}{0.5pt}\end{center}

\hypertarget{rappels}{%
\section{Rappels}\label{rappels}}

\hypertarget{les-verbes-ruxe9guliers}{%
\subsection{Les verbes réguliers}\label{les-verbes-ruxe9guliers}}

\begin{quote}
Les verbes réguliers fonctionnent selon les mêmes principes.

Ils sont appelés ``réguliers'' parce que leur prétérit et leur participe passé se construisent tous deux en ajoutant le suffixe \textless-(e)d\textgreater{} à leur BV :

\begin{itemize}
\item
  \emph{play} \(\rightarrow\) \emph{played, played}
\item
  \emph{decide} \(\rightarrow\) \emph{decided, decided}
\end{itemize}
\end{quote}

\hypertarget{conjugaisons-de-be}{%
\subsection{Conjugaisons de ``BE''}\label{conjugaisons-de-be}}

\begin{column}{0.47\textwidth}

\hypertarget{be-au-pruxe9sent}{%
\subsubsection{``BE'' au présent}\label{be-au-pruxe9sent}}

\begin{table}[H]
\centering
\begin{tabular}{ll}
\toprule
Singulier & Pluriel\\
\midrule
\cellcolor{gray!6}{am} & \cellcolor{gray!6}{are}\\
are & are\\
\cellcolor{gray!6}{is} & \cellcolor{gray!6}{are}\\
\bottomrule
\end{tabular}
\end{table}

\end{column}

\begin{column}{0.06\textwidth}

\hfill\break

\end{column}

\begin{column}{0.47\textwidth}

\hypertarget{be-au-pruxe9tuxe9rit}{%
\subsubsection{``BE'' au prétérit}\label{be-au-pruxe9tuxe9rit}}

\begin{table}[H]
\centering
\begin{tabular}{ll}
\toprule
Singulier & Pluriel\\
\midrule
\cellcolor{gray!6}{was} & \cellcolor{gray!6}{\vphantom{1} were}\\
were & were\\
\cellcolor{gray!6}{was} & \cellcolor{gray!6}{were}\\
\bottomrule
\end{tabular}
\end{table}

\end{column}

\begin{center}\rule{0.5\linewidth}{0.5pt}\end{center}

\hypertarget{tableau-des-verbes-irruxe9guliers}{%
\section{Tableau des verbes irréguliers}\label{tableau-des-verbes-irruxe9guliers}}

\begin{longtable}{>{}lll>{}l}
\toprule
Base Verbale & Prétérit & Participe passé & Traduction\\
\midrule
\cellcolor{gray!6}{\textbf{beat}} & \cellcolor{gray!6}{beat} & \cellcolor{gray!6}{beaten} & \cellcolor{gray!6}{\textbf{battre (coeur)}}\\
\textbf{become} & became & become & \textbf{devenir}\\
\cellcolor{gray!6}{\textbf{begin}} & \cellcolor{gray!6}{began} & \cellcolor{gray!6}{begun} & \cellcolor{gray!6}{\textbf{commencer}}\\
\textbf{bend} & bent & bent & \textbf{(se) courber}\\
\cellcolor{gray!6}{\textbf{bet}} & \cellcolor{gray!6}{bet} & \cellcolor{gray!6}{bet} & \cellcolor{gray!6}{\textbf{parier}}\\
\addlinespace
\textbf{bind} & bound & bound & \textbf{relier (livre)}\\
\cellcolor{gray!6}{\textbf{bite}} & \cellcolor{gray!6}{bit} & \cellcolor{gray!6}{bitten} & \cellcolor{gray!6}{\textbf{mordre}}\\
\textbf{bleed} & bled & bled & \textbf{saigner}\\
\cellcolor{gray!6}{\textbf{blow}} & \cellcolor{gray!6}{blew} & \cellcolor{gray!6}{blown} & \cellcolor{gray!6}{\textbf{souffler}}\\
\textbf{break} & broke & broken & \textbf{casser}\\
\addlinespace
\cellcolor{gray!6}{\textbf{breed}} & \cellcolor{gray!6}{bred} & \cellcolor{gray!6}{bred} & \cellcolor{gray!6}{\textbf{élever}}\\
\textbf{bring} & brought & brought & \textbf{apporter}\\
\cellcolor{gray!6}{\textbf{build}} & \cellcolor{gray!6}{built} & \cellcolor{gray!6}{built} & \cellcolor{gray!6}{\textbf{construire}}\\
\textbf{burn} & burnt & burnt & \textbf{brûler}\\
\cellcolor{gray!6}{\textbf{buy}} & \cellcolor{gray!6}{bought} & \cellcolor{gray!6}{bought} & \cellcolor{gray!6}{\textbf{acheter}}\\
\addlinespace
\textbf{catch} & caught & caught & \textbf{attraper}\\
\cellcolor{gray!6}{\textbf{choose}} & \cellcolor{gray!6}{chose} & \cellcolor{gray!6}{chosen} & \cellcolor{gray!6}{\textbf{choisir}}\\
\textbf{come} & came & come & \textbf{venir}\\
\cellcolor{gray!6}{\textbf{cost}} & \cellcolor{gray!6}{cost} & \cellcolor{gray!6}{cost} & \cellcolor{gray!6}{\textbf{coûter}}\\
\textbf{cut} & cut & cut & \textbf{couper}\\
\addlinespace
\cellcolor{gray!6}{\textbf{do}} & \cellcolor{gray!6}{did} & \cellcolor{gray!6}{done} & \cellcolor{gray!6}{\textbf{faire (auxiliaire)}}\\
\textbf{dig} & dug & dug & \textbf{creuser}\\
\cellcolor{gray!6}{\textbf{draw}} & \cellcolor{gray!6}{drew} & \cellcolor{gray!6}{drawn} & \cellcolor{gray!6}{\textbf{dessiner}}\\
\textbf{dream} & dreamt & dreamt & \textbf{rêver}\\
\cellcolor{gray!6}{\textbf{drink}} & \cellcolor{gray!6}{drank} & \cellcolor{gray!6}{drunk} & \cellcolor{gray!6}{\textbf{boire}}\\
\addlinespace
\textbf{drive} & drove & driven & \textbf{conduire}\\
\cellcolor{gray!6}{\textbf{eat}} & \cellcolor{gray!6}{ate} & \cellcolor{gray!6}{eaten} & \cellcolor{gray!6}{\textbf{manger}}\\
\textbf{fall} & fell & fallen & \textbf{tomber}\\
\cellcolor{gray!6}{\textbf{feed}} & \cellcolor{gray!6}{fed} & \cellcolor{gray!6}{fed} & \cellcolor{gray!6}{\textbf{nourrir}}\\
\textbf{feel} & felt & felt & \textbf{(res)sentir}\\
\addlinespace
\cellcolor{gray!6}{\textbf{fight}} & \cellcolor{gray!6}{fought} & \cellcolor{gray!6}{fought} & \cellcolor{gray!6}{\textbf{se battre}}\\
\textbf{find} & found & found & \textbf{trouver}\\
\cellcolor{gray!6}{\textbf{fly}} & \cellcolor{gray!6}{flew} & \cellcolor{gray!6}{flown} & \cellcolor{gray!6}{\textbf{voler (oiseau)}}\\
\textbf{forget} & forgot & forgotten & \textbf{oublier}\\
\cellcolor{gray!6}{\textbf{forgive}} & \cellcolor{gray!6}{forgave} & \cellcolor{gray!6}{forgiven} & \cellcolor{gray!6}{\textbf{pardonner}}\\
\addlinespace
\textbf{freeze} & froze & frozen & \textbf{geler}\\
\cellcolor{gray!6}{\textbf{get}} & \cellcolor{gray!6}{got} & \cellcolor{gray!6}{got} & \cellcolor{gray!6}{\textbf{obtenir}}\\
\textbf{give} & gave & given & \textbf{donner}\\
\cellcolor{gray!6}{\textbf{go}} & \cellcolor{gray!6}{went} & \cellcolor{gray!6}{gone} & \cellcolor{gray!6}{\textbf{aller}}\\
\textbf{grow} & grew & grown & \textbf{grandir}\\
\addlinespace
\cellcolor{gray!6}{\textbf{have}} & \cellcolor{gray!6}{had} & \cellcolor{gray!6}{had} & \cellcolor{gray!6}{\textbf{avoir}}\\
\textbf{hear} & heard & heard & \textbf{entendre}\\
\cellcolor{gray!6}{\textbf{hide}} & \cellcolor{gray!6}{hid} & \cellcolor{gray!6}{hidden} & \cellcolor{gray!6}{\textbf{cacher}}\\
\textbf{hit} & hit & hit & \textbf{frapper}\\
\cellcolor{gray!6}{\textbf{hold}} & \cellcolor{gray!6}{held} & \cellcolor{gray!6}{held} & \cellcolor{gray!6}{\textbf{tenir}}\\
\addlinespace
\textbf{hurt} & hurt & hurt & \textbf{faire mal}\\
\cellcolor{gray!6}{\textbf{keep}} & \cellcolor{gray!6}{kept} & \cellcolor{gray!6}{kept} & \cellcolor{gray!6}{\textbf{garder}}\\
\textbf{know} & knew & known & \textbf{savoir}\\
\cellcolor{gray!6}{\textbf{lay}} & \cellcolor{gray!6}{laid} & \cellcolor{gray!6}{laid} & \cellcolor{gray!6}{\textbf{poser}}\\
\textbf{lead} & led & led & \textbf{mener}\\
\addlinespace
\cellcolor{gray!6}{\textbf{lean}} & \cellcolor{gray!6}{leant} & \cellcolor{gray!6}{leant} & \cellcolor{gray!6}{\textbf{pencher}}\\
\textbf{leave} & left & left & \textbf{quitter}\\
\cellcolor{gray!6}{\textbf{lend}} & \cellcolor{gray!6}{lent} & \cellcolor{gray!6}{lent} & \cellcolor{gray!6}{\textbf{prêter}}\\
\textbf{let} & let & let & \textbf{laisser}\\
\cellcolor{gray!6}{\textbf{lose}} & \cellcolor{gray!6}{lost} & \cellcolor{gray!6}{lost} & \cellcolor{gray!6}{\textbf{perdre}}\\
\addlinespace
\textbf{make} & made & made & \textbf{fabriquer}\\
\cellcolor{gray!6}{\textbf{mean}} & \cellcolor{gray!6}{meant} & \cellcolor{gray!6}{meant} & \cellcolor{gray!6}{\textbf{signifier}}\\
\textbf{meet} & met & met & \textbf{rencontrer}\\
\cellcolor{gray!6}{\textbf{pay}} & \cellcolor{gray!6}{paid} & \cellcolor{gray!6}{paid} & \cellcolor{gray!6}{\textbf{payer}}\\
\textbf{put} & put & put & \textbf{mettre}\\
\addlinespace
\cellcolor{gray!6}{\textbf{quit}} & \cellcolor{gray!6}{quit} & \cellcolor{gray!6}{quit} & \cellcolor{gray!6}{\textbf{arrêter}}\\
\textbf{read} & read & read & \textbf{lire}\\
\cellcolor{gray!6}{\textbf{ride}} & \cellcolor{gray!6}{rode} & \cellcolor{gray!6}{ridden} & \cellcolor{gray!6}{\textbf{aller en (véhicule)}}\\
\textbf{ring} & rang & rung & \textbf{sonner}\\
\cellcolor{gray!6}{\textbf{rise}} & \cellcolor{gray!6}{rose} & \cellcolor{gray!6}{risen} & \cellcolor{gray!6}{\textbf{monter}}\\
\addlinespace
\textbf{run} & ran & run & \textbf{courir}\\
\cellcolor{gray!6}{\textbf{say}} & \cellcolor{gray!6}{said} & \cellcolor{gray!6}{said} & \cellcolor{gray!6}{\textbf{dire}}\\
\textbf{see} & saw & seen & \textbf{voir}\\
\cellcolor{gray!6}{\textbf{sell}} & \cellcolor{gray!6}{sold} & \cellcolor{gray!6}{sold} & \cellcolor{gray!6}{\textbf{vendre}}\\
\textbf{send} & sent & sent & \textbf{envoyer}\\
\addlinespace
\cellcolor{gray!6}{\textbf{set}} & \cellcolor{gray!6}{set} & \cellcolor{gray!6}{set} & \cellcolor{gray!6}{\textbf{régler (machine)}}\\
\textbf{shake} & shook & shaken & \textbf{secouer}\\
\cellcolor{gray!6}{\textbf{shine}} & \cellcolor{gray!6}{shone} & \cellcolor{gray!6}{shone} & \cellcolor{gray!6}{\textbf{briller}}\\
\textbf{shoe} & shod & shod & \textbf{chausser (rare)}\\
\cellcolor{gray!6}{\textbf{shoot}} & \cellcolor{gray!6}{shot} & \cellcolor{gray!6}{shot} & \cellcolor{gray!6}{\textbf{tirer}}\\
\addlinespace
\textbf{show} & showed & shown & \textbf{montrer}\\
\cellcolor{gray!6}{\textbf{shrink}} & \cellcolor{gray!6}{shrank} & \cellcolor{gray!6}{shrunk} & \cellcolor{gray!6}{\textbf{rétrécir}}\\
\textbf{shut} & shut & shut & \textbf{fermer}\\
\cellcolor{gray!6}{\textbf{sing}} & \cellcolor{gray!6}{sang} & \cellcolor{gray!6}{sung} & \cellcolor{gray!6}{\textbf{chanter}}\\
\textbf{sink} & sank & sunk & \textbf{couler}\\
\addlinespace
\cellcolor{gray!6}{\textbf{sit}} & \cellcolor{gray!6}{sat} & \cellcolor{gray!6}{sat} & \cellcolor{gray!6}{\textbf{s'asseoir}}\\
\textbf{sleep} & slept & slept & \textbf{dormir}\\
\cellcolor{gray!6}{\textbf{speak}} & \cellcolor{gray!6}{spoke} & \cellcolor{gray!6}{spoken} & \cellcolor{gray!6}{\textbf{parler}}\\
\textbf{spend} & spent & spent & \textbf{dépenser}\\
\cellcolor{gray!6}{\textbf{spill}} & \cellcolor{gray!6}{spilt} & \cellcolor{gray!6}{spilt} & \cellcolor{gray!6}{\textbf{renverser}}\\
\addlinespace
\textbf{spread} & spread & spread & \textbf{répandre}\\
\cellcolor{gray!6}{\textbf{speed}} & \cellcolor{gray!6}{sped} & \cellcolor{gray!6}{sped} & \cellcolor{gray!6}{\textbf{aller très/trop vite}}\\
\textbf{stand} & stood & stood & \textbf{se tenir debout}\\
\cellcolor{gray!6}{\textbf{steal}} & \cellcolor{gray!6}{stole} & \cellcolor{gray!6}{stolen} & \cellcolor{gray!6}{\textbf{voler (crime)}}\\
\textbf{stick} & stuck & stuck & \textbf{planter}\\
\addlinespace
\cellcolor{gray!6}{\textbf{sting}} & \cellcolor{gray!6}{stung} & \cellcolor{gray!6}{stung} & \cellcolor{gray!6}{\textbf{piquer}}\\
\textbf{stink} & stank & stunk & \textbf{puer}\\
\cellcolor{gray!6}{\textbf{swear}} & \cellcolor{gray!6}{swore} & \cellcolor{gray!6}{sworn} & \cellcolor{gray!6}{\textbf{jurer}}\\
\textbf{sweep} & swept & swept & \textbf{balayer}\\
\cellcolor{gray!6}{\textbf{swim}} & \cellcolor{gray!6}{swam} & \cellcolor{gray!6}{swum} & \cellcolor{gray!6}{\textbf{nager}}\\
\addlinespace
\textbf{swing} & swung & swung & \textbf{osciller}\\
\cellcolor{gray!6}{\textbf{take}} & \cellcolor{gray!6}{took} & \cellcolor{gray!6}{taken} & \cellcolor{gray!6}{\textbf{prendre}}\\
\textbf{teach} & taught & taught & \textbf{enseigner}\\
\cellcolor{gray!6}{\textbf{tear}} & \cellcolor{gray!6}{tore} & \cellcolor{gray!6}{torn} & \cellcolor{gray!6}{\textbf{déchirer}}\\
\textbf{tell} & told & told & \textbf{raconter}\\
\addlinespace
\cellcolor{gray!6}{\textbf{think}} & \cellcolor{gray!6}{thought} & \cellcolor{gray!6}{thought} & \cellcolor{gray!6}{\textbf{penser}}\\
\textbf{throw} & threw & thrown & \textbf{jeter}\\
\cellcolor{gray!6}{\textbf{understand}} & \cellcolor{gray!6}{understood} & \cellcolor{gray!6}{understood} & \cellcolor{gray!6}{\textbf{comprendre}}\\
\textbf{wake} & woke & woken & \textbf{réveiller}\\
\cellcolor{gray!6}{\textbf{wear}} & \cellcolor{gray!6}{wore} & \cellcolor{gray!6}{worn} & \cellcolor{gray!6}{\textbf{porter (vêtement)}}\\
\addlinespace
\textbf{win} & won & won & \textbf{gagner}\\
\cellcolor{gray!6}{\textbf{write}} & \cellcolor{gray!6}{wrote} & \cellcolor{gray!6}{written} & \cellcolor{gray!6}{\textbf{écrire}}\\
\bottomrule
\end{longtable}

%\end{linenumbers}

\end{document}
